\documentclass[12pt,titlepage]{article}
\usepackage[utf8]{inputenc}
\usepackage[T1]{fontenc}
\usepackage[english]{babel}
\usepackage[a4paper]{geometry}
\frenchspacing
\usepackage{amsfonts}
\usepackage{amsmath}
\usepackage{amssymb}
\usepackage{amsthm}
\usepackage{setspace}
\usepackage{fullpage}
\usepackage{tocbibind}
\usepackage{graphicx}
\usepackage{url}
\usepackage{verbatim}
\usepackage{listings}
%\usepackage{gitinfo2}
\usepackage{hyperref}
\usepackage{cleveref}
\usepackage{cite}
\usepackage{color}
\usepackage{enumitem}
\usepackage{usecases}
\usepackage{nameref}

\definecolor{pblue}{rgb}{0.13,0.13,1}
\definecolor{pgreen}{rgb}{0,0.5,0}
\definecolor{pred}{rgb}{0.9,0,0}
\definecolor{pgrey}{rgb}{0.46,0.45,0.48}
\lstset{language=Java,
  showspaces=false,
  showtabs=false,
  breaklines=true,
  showstringspaces=false,
  breakatwhitespace=true,
  commentstyle=\color{pgreen},
  keywordstyle=\color{pblue},
  stringstyle=\color{pred},
  basicstyle=\ttfamily,
  moredelim=[il][\textcolor{pgrey}]{$$},
  moredelim=[is][\textcolor{pgrey}]{\%\%}{\%\%}
}
\renewcommand{\labelitemi}{$\bullet$}
\renewcommand{\labelitemii}{$\cdot$}
\renewcommand{\labelitemiii}{$\diamond$}
\renewcommand{\labelitemiv}{$\ast$}


\begin{document}
\title{
	Design Document for Bicycle Garage Pro\\
	(Group 33, 2015)\\
	\vspace{0.2in}
	\normalsize Current version: 0.9.1
}
\author{
	Alexander Skafte\\
	\url{tfy13ask@student.lu.se}\\
	Dennis Jin\\
	\url{desuvader@gmail.com}\\
	Petter Berntsson\\
	\url{dat14pbe@student.lu.se}\\
	Emelie Löthman\\
	\url{pol14elo@student.lu.se}\\
	Adam Mzrozek\\
	\url{dat14amr@student.lu.se}
}
\date{}



\maketitle
\newpage
\tableofcontents
\thispagestyle{empty}
\setcounter{page}{0}
\newpage


% ----- REFERENCES & -----------------------------------------------------------
% ----- BIBLIOGRAPHY -----------------------------------------------------------

\section{References}
\label{sec:references}

\begin{itemize}
	\item \textit{Examples and Exercises in the Software
		Engineering Process}. ETSA01 VT 2015. Deparment of Computer
		Science, Lund University. March 10, 2015.
	\item \textit{Requirements Specification for Bicycle Garage
		Pro}. ETSA01, Group 33, 2015.
\end{itemize}


% ----- INTRODUCTION -----------------------------------------------------------

\section{Introduction}

\subsection{Project model}

\subsection{Purpose}

The aim of the project is to develop software for a bicycle garage which
provides safe storage of bicycles. The software will be developed for
prespecified hardware.

\subsection{Goals}

\subsubsection{Product goals}

The software shall be able to securely handle users checking in and out their
bicycles of the garage. 

The product will take care of users passing through the entrance with their
bicycle - the user will enter a PIN code and scan a barcode for their bicycle to
store it. The user can then exit the garage, and to exit with a bicycle the same
procedure will be done as the one to store it.

For a more detailed overview of the product, please refer to
\textit{Requirements Specification for Bicycle Garage Pro}.

\subsubsection{Business goals}

To offer cheap and reliable bicycle storage to customers, whom may want to
protect their bicycle against misfortune and use the bicycle instead of public
transport of their cars.

\subsection{Limitations}

\begin{description}
\item[Hardware limitations] \hfill \\ 
Since the software must make use of prespecified hardware, there is a limit set
by factors other than that of the software's implementation that will decide
how effective the system will be at doing its job. If there are fundamental
flaws in the hardware, the software too will suffer. To counter this potential
issue, thorough testing must be implemented in the software.

\item[Temporal limitations] \hfill \\
In addition, there is a clear deadline for the project. This may result in an
unfinished and/or underdeveloped product being released, which may result in
expensive bugs and a bad reputation should the software fail its customers.

\item[Economical limitations] \hfill \\
The budget is limited, and thus more developers or other personnel cannot be
hired.
\end{description}


% ----- PROJECT ORGANIZATION ---------------------------------------------------

\section{Project Organization}

\subsection{Development organization}

Responsible for for this project are:

\begin{description}
	\item[Alexander Skafte:] \hfill
		\begin{itemize}
			\item Requirements specification
			\item Test plan
			\item Design document
			\item Project plan
			\item Various reviews
			\item Software development
		\end{itemize}
	\item[Dennis Jin:] \hfill
		\begin{itemize}
			\item Requirements specification
			\item Test plan
			\item Design document
			\item Various reviews
			\item Software development
		\end{itemize}
	\item[Petter Berntsson:] \hfill
		\begin{itemize}
			\item Project plan
		\end{itemize}
\end{description}

\subsection{Stakeholders}

\begin{itemize}
	\item \textbf{The municipality} -- For environmental reasons; if an
		increased number of citizens use their bicycles instead of their
		cars or public transport, the greenhouse effect can be lessened.
	\item \textbf{ACME} -- The company which manufactures the bicycle
		garages.
\end{itemize}


% ----- HARDWARE AND SOFTWARE RESOURCES  ---------------------------------------

\section{Hardware and software resources}
\label{sec:hardware-and-software-resources}

\begin{description}
	\item[\LaTeX:] Typesetting software; used to typeset all documents
		related to the development of \textit{Bicycle Garage Pro}.
	\item[Eclipse:] Integrated desktop environment for software development.
	\item[The Java Programming Language:] The programming language used to
		create the software.
	\item[Google Drive:] Cloud storage reachable for all members to edit.
	\item[Git \& GitHub:] For sharing code and \LaTeX~source text.
\end{description}


% ----- DIVISION OF LABOR ------------------------------------------------------

\section{Division of labor}

\subsection{Activities}

\begin{description}
	\item[Project plan] \hfill \\
		The planning of the project; also this rapport.
	\item[Requirements specification] \hfill \\
		The requirements for the project are identified and defined.
		This results in a requirements specification.
	\item[Test plan] \hfill \\
		The planning of all tests that will be performed.
		This results in a test plan.
	\item[Design] \hfill \\
		A general description of the systems structure.
		This results in a design document.
		Different people are responsible for different parts of the
		code.
	\item[Implementation and unit testing] \hfill \\
		Implementation and testing of all parts of the system.
		This is performed according to the design.
	\item[Integration] \hfill \\
		This part is the finishing part of the program,
		which leads to a fully functioning program.
	\item[System test] \hfill \\
		The system is tested in its entirety. This takes part in a
		testing environment which mimics the environment that the
		program will be used in.
\end{description}

\subsection{Deliverables}

\begin{itemize}
	\item Project plan
	\item Requirements specification
	\item Design document
	\item Test plan
	\item Source code
	\item Runnable JAR file
\end{itemize}

\subsection{Schedule and estimated work}

Below is a schedule showing the time estimated for the project's different
parts. The colors carry no significance.

\begin{figure}[H]
	\centering
		\includegraphics[width=1.0\textwidth]{assets/schedule}
\end{figure}


% ----- REPORT, FOLLOW-UP AND QUALITY ASSURANCE --------------------------------

\section{Report, follow-up and quality assurance}

To make collaboration easy when writing both the software and the project
related documents, the version control system \textit{Git} will be used, with
the files hosted on the free Git hosting platform \textit{GitHub}. In addition,
both internal and external reviews will be conducted in a common \textit{Google
Drive} folder, using the \textit{Google Docs} service.

All PDF versions of the documents will be found in the Google Drive folder when
they need to be there. The latest versions shall always be available on GitHub.

During the seminar sessions provided, the group will gather and discuss further
plans for the project, as well as meet with the tutor who will provide feedback
on the documents handed in through the Google Drive folder.

Communication within the group will be conducted through a social media platform
previously agreed on, or through email if necessary. The reason for primarily
using a social media platform for this purpose is that communication is faster
and more interactive than if one were to solely use emails. Additionally, commit
messages on Git shall be clear and provide a third means of communication.


% ----- APPENDIX ---------------------------------------------------------------
% ----- RISK ANALYSIS ----------------------------------------------------------

\newpage
\appendix

\section{Risk analysis}
\label{app:risk-analysis}

\begin{usecase}
	\addtitle{Risk case 1:}{The hardware does not comply with its
		specification.}
	\addfield{Risk:}{Low.}
	\addfield{Effect:}{Devastating.}
	\addfield{Measure:}{Contact the hardware developers and ask for new
		hardware specifications.}
	\addfield{Risk indicator:}{The appliance does not work properly, even if
		the virtual testing passed.}
\end{usecase}

\begin{usecase}
	\addtitle{Risk case 2:}{Unwanted document modification or deletion}
	\addfield{Risk:}{Possible}
	\addfield{Effect:}{Potentially devastating}
	\addfield{Measure:}{Use a version control system (such as Git).}
	\addfield{Risk indicator:}{Document metadata contains incorrect dates
		and/or versions.}
\end{usecase}

\begin{usecase}
	\addtitle{Risk case 3:}{Personnel absence due to diseases or other
		obstacles.}
	\addfield{Risk:}{Possible.}
	\addfield{Effect:}{Moderate or potentially high.}
	\addfield{Measure:}{Restructure work distribution.}
	\addfield{Risk indicator:}{Group members stop attending seminars and/or
		stop responding on the specified communication channel(s).}
\end{usecase}

\begin{usecase}
	\addtitle{Risk case 4:}{The client modifies the product requirements.}
	\addfield{Risk:}{Low.}
	\addfield{Effect:}{Moderate--depends on how large the modification is.}
	\addfield{Measure:}{Revise the product plan.}
	\addfield{Risk indicator:}{-}
\end{usecase}


%% Template for risk cases:
%	\begin{usecase}
%		\addtitle{Risk case #:}{}
%		\addfield{Risk:}{}
%		\addfield{Effect:}{}
%		\addfield{Measure:}{}
%		\addfield{Risk indicator:}{}
%	\end{usecase}
%	

\end{document}

