\documentclass[12pt,titlepage,bibliography=totoc]{article}
\usepackage[utf8]{inputenc}
\usepackage[T1]{fontenc}
\usepackage[english]{babel}
\usepackage[a4paper]{geometry}
\frenchspacing
\usepackage{amsfonts}
\usepackage{amsmath}
\usepackage{amssymb}
\usepackage{amsthm}
\usepackage{setspace}
\usepackage{fullpage}
\usepackage{tocbibind}
\usepackage{graphicx}
\usepackage{url}
\usepackage{verbatim}
\usepackage{listings}
%\usepackage{gitinfo2}
\usepackage{hyperref}
\usepackage{cleveref}
\usepackage{cite}
\usepackage{color}
\usepackage{enumitem}
\usepackage{usecases}
\usepackage{nameref}

\definecolor{pblue}{rgb}{0.13,0.13,1}
\definecolor{pgreen}{rgb}{0,0.5,0}
\definecolor{pred}{rgb}{0.9,0,0}
\definecolor{pgrey}{rgb}{0.46,0.45,0.48}
\lstset{language=Java,
  showspaces=false,
  showtabs=false,
  breaklines=true,
  showstringspaces=false,
  breakatwhitespace=true,
  commentstyle=\color{pgreen},
  keywordstyle=\color{pblue},
  stringstyle=\color{pred},
  basicstyle=\ttfamily,
  moredelim=[il][\textcolor{pgrey}]{$$},
  moredelim=[is][\textcolor{pgrey}]{\%\%}{\%\%}
}
\renewcommand{\labelitemi}{$\bullet$}
\renewcommand{\labelitemii}{$\cdot$}
\renewcommand{\labelitemiii}{$\diamond$}
\renewcommand{\labelitemiv}{$\ast$}


% TODO:
%	Every subtitle will have the following stuff:
%		- Performed by:
%		- Type of test:
%		- Criterion
%		- Stop rule

% TODO:
%	Remove empty newlines at the bottom of this document :)

\begin{document}
\title{
	Design Document for Bicycle Garage Pro\\
	(Group 33, 2015)\\
	\vspace{0.2in}
	\normalsize Current version: 0.9.1
}
\author{
	Alexander Skafte\\
	\url{tfy13ask@student.lu.se}\\
	Dennis Jin\\
	\url{desuvader@gmail.com}\\
	Petter Berntsson\\
	\url{dat14pbe@student.lu.se}\\
	Emelie Löthman\\
	\url{pol14elo@student.lu.se}\\
	Adam Mzrozek\\
	\url{dat14amr@student.lu.se}
}
\date{}



\maketitle
\newpage
\tableofcontents
\thispagestyle{empty}
\setcounter{page}{0}
\newpage

% ----- REFERENCES -------------------------------------------------------------

\section{References}
\label{sec:references}

% ----- INTRODUCTION -----------------------------------------------------------

\section{Introduction}
\label{sec:introduction}
\subsection{Tested system}

The system described in this document is the software for a public bicycle
garage. This software is responsible for managing the authentication of users
and the management of their information and their bicycles.

This document provides a specification for testing the bicycle garage software.
The test process consists of the following phases:

\begin{itemize}
	\item Unit testing
	\item Integration testing
	\item System testing
	\item Acceptance testing
\end{itemize}

% ----- TEST PROCESS -----------------------------------------------------------

\section{Test process}
\label{sec:test-process}
\subsection{Process overview}


\subsection{Unit testing}
\label{subsec:unit-testing}

Every non-trivial function is tested in software through the use of a test suite
library.

\begin{description}
	\item[Performed by:]	Developers
	\item[Type of test:]	Structural
	\item[Criteria:]	Every line of code is tested
	\item[Stop rule:]	No errors found
\end{description}


\subsection{Integration testing}
\label{subsec:integration-testing}

Integration testing is performed in a similar way to unit testing, but larger
and more inclusive modules are tested. Each module is tested in software through
the use of a test suite library.

\begin{description}
	\item[Performed by:]	Developers
	\item[Type of test:]	Structural
	\item[Criteria:]	Every API method is tested completely
	\item[Stop rule:]	No errors found
\end{description}


\subsection{System testing}
\label{subsec:system-testing}

During system testing, all requirements specified inside the Software
Requirements Specification are tested.

\begin{description}
	\item[Performed by:]	Developers
	\item[Type of test:]	Functional
	\item[Criteria:]	All requirements inside the SRS are fulfilled
	\item[Stop rule:]	No critical errors found
\end{description}

\subsection{Acceptance testing}

Acceptance testing is performed by the client and not the developers, and is
therefore not discussed in this document.

% ----- TESTED ITEMS -----------------------------------------------------------

\section{Tested items}
\label{sec:tested-items}

%
% TODO TODO TODO TODO TODO TODO TODO TODO TODO TODO TODO TODO TODO TODO TODO
%

% ----- TEST RECORDING PROCEDURE -----------------------------------------------

\section{Test recording procedure}
\label{sec:test-recording-procedure}
\subsection{Unit testing}
\subsection{Integration testing}
\subsection{System testing}
\subsection{Acceptance testing}

% ----- TEST CASES FOR SYSTEM TESTING ------------------------------------------

\section{Test cases for system testing}
\label{sec:test-cases-for-system-testing}
\subsection{Test cases}
\subsection{Requirements coverage and traceability}

\bibliography{bibliography}{}
\bibliographystyle{plain}

% TODO: Fix this shit later
%\appendix

\section{Hardware API}
\label{app:hardware}
This API specification is directly copied from \url{http://cs.lth.se/etsa01/projekt-2015/haardvarugraenssnitt-och-drivrutiner/}. If there are any questions regarding the hardware API, please contact Markus Borg (\url{Markus.Borg@cs.lth.se}).

\begin{lstlisting}[language=Java]
public interface BarcodePrinter {
	/* Print a bicycleID as a barcode.
	 * Bicycle ID should be a string of 5 characters, where every
	 * character can be '0', '1',... '9'. */

	public void printBarcode(String bicycleID);
}

public interface PinCodeTerminal { 
	/* Register bicycle garage manager so 
	 * that the pin code terminal knows 
	 * which manager to call when a user has 
	 * pressed a key. */ 

	public void register(BicycleGarageManager manager); 

	/* Turn on LED for lightTime seconds. 
	 * Colour: 
	 * colour = RED_LED = 0 => red 
	 * colour = GREEN_LED = 1 => green */ 

	public void lightLED(int colour, int lightTime); 
	public static final int RED_LED = 0, 
	GREEN_LED = 1;
}

public interface BarcodeReader {     
	/* Register bicycle garage manager 
	 * so that the bar code reader knows 
	 * which manager to call when a user 
	 * has used the reader. */  

	public void register(BicycleGarageManager manager); 
}

public interface ElectronicLock {
	/* Open the lock for timeOpen seconds.  */

	public void open(int timeOpen);
}
\end{lstlisting}

% ----- APPENDIX ---------------------------------------------------------------

\newpage
\appendix

\section{Test cases}
\label{app:test-cases}

\begin{usecase}
	\addtitle{Test case 1:}{Registration of a new user}
	\addfield{Primary actor:}{Operator}
	\addfield{Preconditions:}{User is unregistered}
	\addfield{Postconditions:}{User is registered}
	\addscenario{Main success scenario:}{
		\item Operator provides the required user information to the
			control interface.
		\item A new PIN code is generated for the user.
		\item The user is added to the system.
	}
\end{usecase}

\begin{usecase}
	\addtitle{Test case 2:}{Registration of an already registered user}
	\addfield{Primary actor:}{Operator}
	\addfield{Preconditions:}{User is registered}
	\addfield{Postconditions:}{User is registered}
	\addscenario{Main success scenario:}{
		\item Operator provides the required user information to the
			control interface.
		\item The system responds with an error message, e.g. ''The user
			is already registered.''
	}
\end{usecase}

\begin{usecase}
	\addtitle{Test case 3:}{Unregistration of a registered user}
	\addfield{Primary actor:}{Operator}
	\addfield{Preconditions:}{User is registered}
	\addfield{Postconditions:}{User is unregistered}
	\addscenario{Main success scenario:}{
		\item Operator provides the required user information to the
			control interface.
		\item All bicycles associated with the user are removed from the
			system.
		\item The user is removed from the system.
	}
\end{usecase}

\begin{usecase}
	\addtitle{Test case 4:}{Association of a new bicycle with a user}
	\addfield{Primary actor:}{Operator}
	\addfield{Preconditions:}{User is registered; garage is not full}
	\addfield{Postconditions:}{Bicycle is associated with user}
	\addscenario{Main success scenario:}{
		\item Operator provides the required user information to the
			control interface.
		\item A unique 5-digit identification number is generated and
			associated with the bicycle.
		\item The bicycle is added to the set of bicycles owned by the
			user.
		\item A barcode associated with the 5-digit ID is printed and
			given to the user.
	}
\end{usecase}

\begin{usecase}
	\addtitle{Test case 5:}{Disassociation of a user's bicycle}
	\addfield{Primary actor:}{Operator}
	\addfield{Preconditions:}{User is registered; bicycle is associated with
					user.}
	\addfield{Postconditions:}{Bicycle is not associated with user nor is it
					present in the system.}
	\addscenario{Main success scenario:}{
		\item Operator provides the required user information to the
			control interface.
		\item The bicycle is disassociated with the user.
		\item The unique 5-digit identification number associated with
			the bicycle is returned to the pool of available ID's.
			As a consequence, the barcode is rendered invalid.
	}
\end{usecase}

\end{document}


















